\begin{titlepage}

%%%
\begin{center}
{\large \sf WHITE PAPER} \\ \vspace{0.65cm}
{\Huge \bf Towards a Barium Experiment in a Xenon TPC (BEXT)} \\ \vspace{0.75cm}
{(16 May 2011)} \\ \vspace{0.35cm}
\end{center}
%%%

\begin{center}
\begin{minipage}{15cm}
\section{Introduction}
We outline in this paper the theoretical basis and our proposed experimental roadmap to study the feasibility of tagging the single barium ion produced in the  \bb\ decay of \XE. The possibility of resonant excitation of the barium ion to its first excited state using lasers of suitable wavelength and the subsequent observation ``tagging" of the emitted fluorescence was proposed by XXX and has been actively pursued, for more than XXX years by members of the EXO collaboration, which have conducted a very active R\&D, considering tagging both in liquid and gaseous xenon TPCs.  

Both EXO and its proposed continuation, NEXO, are liquid xenon TPCs (LXe). The NEXT collaboration, on the other hand, is currently building the NEXT-100 detector, a high pressure xenon TPC (HPXe) using electroluminescence to amplify the ionisation signal. The R\&D phase of NEXT, already completed, has shown excellent energy resolution and the capability of reconstruction of electron tracks. 

Barium tagging in LXe and HPXe are two different problems. The R\&D of EXO has been focused on extracting the barium ion, either using a probe (in the LXe) detector, or a dynamic electric field to guide the ion outside a gas TPC. In both cases, the idea is to confine the barium ion in a magneto trap in vacuum, before shining the lasers. The rational behind this approach is clear: in situ tagging, even it one demonstrates that is possible, would be, in any case, difficult. It would require good and fast location of the ion, a reliable laser guiding system, and sophisticated instrumentation. On the other hand it may turn out that extracting the ion from the chamber is an even more formidable problem. 

The authors of this paper are members of the NEXT collaboration and of the CLPU. Our goal is to outline the theoretical basis and the experimental roadmap to establish if barium tagging in situ in a HPXE TPC if possible.  

\end{minipage}
\end{center}

\begin{center}
\pagebreak 

{\LARGE \bf The BEXT collaboration}

\vspace{0.4cm}

{\small \sc . Peralta Conde\symbolfootnote[1]{Contact email: aperalta@clpu.es}, J. Api\~naniz, E. Garc\'ia,  M. Rico, C. Salgado, M. S\'anchez, F. Valle, A. Vazquez, and L. Roso}\\
{\it Centro de L\'aseres Pulsados, CLPU, Parque Cient\'ifico, E-37185 Villamayor, Salamanca, Spain.}

\vspace{0.3cm}

{\small \sc  J.J.~G\'omez-Cadenas\symbolfootnote[1]{Contact email: gomez@mail.cern.ch}, P. Ferrario, D. Gonz\'alez, A. Goldschmidt N.~L\'opez-March,  \\I.~Liubarsky, D. Nygren, F.~Monrabal, A. ~Simó}\\
{\it Instituto de F\'isica Corpuscular (IFIC), CSIC \& Univ.\ de Valencia, Valencia, Spain}

\end{center}




\end{titlepage}
